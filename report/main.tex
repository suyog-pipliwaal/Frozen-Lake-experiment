\documentclass[11pt]{article}    
\usepackage[loose,nice]{units} %replace "nice" by "ugly" for units in upright fractions
\usepackage{mathtools}
\DeclarePairedDelimiter\ceil{\lceil}{\rceil}
\DeclarePairedDelimiter\floor{\lfloor}{\rfloor}
\usepackage{amsmath,amsfonts,stmaryrd,amssymb} % Math packages
\usepackage{enumerate} % Custom item numbers for enumerations
\usepackage{enumitem}
\usepackage[ruled]{algorithm2e} % Algorithms
\usepackage[framemethod=tikz]{mdframed} % Allows defining custom boxed/framed environments
\usepackage{listings} % File listings, with syntax highlighting
\lstset{
	basicstyle=\ttfamily, % Typeset listings in monospace font
}
\usepackage{subfig}
%----------------------------------------------------------------------------------------
%	DOCUMENT MARGINS
%----------------------------------------------------------------------------------------

\usepackage{geometry} % Required for adjusting page dimensions and margins
\geometry{
	paper=a4paper, % Paper size, change to letterpaper for US letter size
	top=2.5cm, % Top margin
	bottom=3cm, % Bottom margin
	left=2.5cm, % Left margin
	right=2.5cm, % Right margin
	headheight=14pt, % Header height
	footskip=1.5cm, % Space from the bottom margin to the baseline of the footer
	headsep=1.2cm, % Space from the top margin to the baseline of the header
	%showframe, % Uncomment to show how the type block is set on the page
}

%----------------------------------------------------------------------------------------
%	FONTS
%----------------------------------------------------------------------------------------

%----------------------------------------------------------------------------------------
%	COMMAND LINE ENVIRONMENT
%----------------------------------------------------------------------------------------

% Usage:
% \begin{commandline}
%	\begin{verbatim}
%		$ ls
%		
%		Applications	Desktop	...
%	\end{verbatim}
% \end{commandline}

\mdfdefinestyle{commandline}{
	leftmargin=10pt,
	rightmargin=10pt,
	innerleftmargin=15pt,
	middlelinecolor=black!50!white,
	middlelinewidth=2pt,
	frametitlerule=false,
	backgroundcolor=black!5!white,
	frametitle={Command Line},
	frametitlefont={\normalfont\sffamily\color{white}\hspace{-1em}},
	frametitlebackgroundcolor=black!50!white,
	nobreak,
}

% Define a custom environment for command-line snapshots
\newenvironment{commandline}{
	\medskip
	\begin{mdframed}[style=commandline]
}{
	\end{mdframed}
	\medskip
}

%----------------------------------------------------------------------------------------
%	FILE CONTENTS ENVIRONMENT
%----------------------------------------------------------------------------------------

% Usage:
% \begin{file}[optional filename, defaults to "File"]
%	File contents, for example, with a listings environment
% \end{file}

\mdfdefinestyle{file}{
	innertopmargin=1.6\baselineskip,
	innerbottommargin=0.8\baselineskip,
	topline=false, bottomline=false,
	leftline=false, rightline=false,
	leftmargin=2cm,
	rightmargin=2cm,
	singleextra={%
		\draw[fill=black!10!white](P)++(0,-1.2em)rectangle(P-|O);
		\node[anchor=north west]
		at(P-|O){\ttfamily\mdfilename};
		%
		\def\l{3em}
		\draw(O-|P)++(-\l,0)--++(\l,\l)--(P)--(P-|O)--(O)--cycle;
		\draw(O-|P)++(-\l,0)--++(0,\l)--++(\l,0);
	},
	nobreak,
}

% Define a custom environment for file contents
\newenvironment{file}[1][File]{ % Set the default filename to "File"
	\medskip
	\newcommand{\mdfilename}{#1}
	\begin{mdframed}[style=file]
}{
	\end{mdframed}
	\medskip
}

%----------------------------------------------------------------------------------------
%	NUMBERED QUESTIONS ENVIRONMENT
%----------------------------------------------------------------------------------------

% Usage:
% \begin{question}[optional title]
%	Question contents
% \end{question}

\mdfdefinestyle{question}{
	innertopmargin=1.2\baselineskip,
	innerbottommargin=0.8\baselineskip,
	roundcorner=5pt,
	nobreak,
	singleextra={%
		\draw(P-|O)node[xshift=1em,anchor=west,fill=white,draw,rounded corners=5pt]{%
		Question \theQuestion\questionTitle};
	},
}

\newcounter{Question} % Stores the current question number that gets iterated with each new question

% Define a custom environment for numbered questions
\newenvironment{question}[1][\unskip]{
	\bigskip
	\stepcounter{Question}
	\newcommand{\questionTitle}{~#1}
	\begin{mdframed}[style=question]
}{
	\end{mdframed}
	\medskip
}

%----------------------------------------------------------------------------------------
%	WARNING TEXT ENVIRONMENT
%----------------------------------------------------------------------------------------

% Usage:
% \begin{warn}[optional title, defaults to "Warning:"]
%	Contents
% \end{warn}

\mdfdefinestyle{warning}{
	topline=false, bottomline=false,
	leftline=false, rightline=false,
	nobreak,
	singleextra={%
		\draw(P-|O)++(-0.5em,0)node(tmp1){};
		\draw(P-|O)++(0.5em,0)node(tmp2){};
		\fill[black,rotate around={45:(P-|O)}](tmp1)rectangle(tmp2);
		\node at(P-|O){\color{white}\scriptsize\bf !};
		\draw[very thick](P-|O)++(0,-1em)--(O);%--(O-|P);
	}
}

% Define a custom environment for warning text
\newenvironment{warn}[1][Warning:]{ % Set the default warning to "Warning:"
	\medskip
	\begin{mdframed}[style=warning]
		\noindent{\textbf{#1}}
}{
	\end{mdframed}
}

%----------------------------------------------------------------------------------------
%	INFORMATION ENVIRONMENT
%----------------------------------------------------------------------------------------

% Usage:
% \begin{info}[optional title, defaults to "Info:"]
% 	contents
% 	\end{info}

\mdfdefinestyle{info}{%
	topline=false, bottomline=false,
	leftline=false, rightline=false,
	nobreak,
	singleextra={%
		\fill[black](P-|O)circle[radius=0.4em];
		\node at(P-|O){\color{white}\scriptsize\bf i};
		\draw[very thick](P-|O)++(0,-0.8em)--(O);%--(O-|P);
	}
}

% Define a custom environment for information
\newenvironment{info}[1][Info:]{ % Set the default title to "Info:"
	\medskip
	\begin{mdframed}[style=info]
		\noindent{\textbf{#1}}
}{
	\end{mdframed}
}




\title{ECS7002P Artificial Intelligence in Games: Assignment 2} % Title of the assignment

\author{Suyog Pipliwal \texttt{210634338}\\ 
			Paul Mutawe \texttt{210715187} \\ 
			Animesh Devendra Chourey \texttt{210765551}
} % Author name and email address

\date{Queen Mary University of London --- \today} % University, school and/or department name(s) and a date

\begin{document}
	\maketitle
\begin{abstract}
	This report is a reflection of work done for group assignment-2 for module ECS7002P. The objective of this assignment is to implement tabular and non-tabular algorithms for finding optimal policy for the given environment. 
\end{abstract}
\section{introduction}
	\begin{enumerate}
		\item Envoirmonemtn
		\item Tabular bases
			\
	\end{enumerate}
\section {Question}
	\begin{enumerate}
		\item 
		
		\item For the policy iteration to find the optimal policy for the big frozen lake, the number of iterations it required are ----. Whereas, for the value iteration the number of iterations required are ----. Since the ---- iteration algorithm required less number of iterations it was faster as compared to the  ----- iteration algorithm.
		
		\item Under the same commom conditions, both converge to the real value functions but at different rates. Sarsa contol require ---- episodes to find the optimal policy whereas Q-learning control needed ---- episodes to find the optimal policy for the the small frozen lake. 
		
		\item Number of features are calculated by number of states * number of actions.These features are needed to calculate the dimensions of $\phi$(s,a) vector as it's dimensions are number of actions by number of features. Parameter vector $\theta$ has same dimensions as number of features. One hot encoded feature vector $\phi$(s,a) can be used as a selector using dot product to filter out the values of the unrelated (state,action) pair. Whike training the parameter vector $\theta$ can be updated to again utilize the feature vector to correctly update the encoded (state,value) and not altering the unrelated (state,action) values. \par
		
		Non tablular model free algorithm does not hold every (state,action) value and the policy in the memory, when the environment is finite. It just encodes the state in question without interrupting the run of the program. Parameter vector $\theta$ stores the general information regarding what values of the states and policies that must be taken. This can be decoded later to get the actual (state,action) values for the state using one hot encoded feature vector. Whereas the tablular mopdel free algorithm is a special case of non tabular model free algorithm becuase the table is created beforehand. It is a special form becuase the encoded values are individually kept in the memory.
		
		\item We were not able to find optimal policy for the Sarsa control and the Q-learning control for the big frozen lake. THis could be becuase the space is too large. Both the Sarsa and Q-learning take the updates the values of the path which they take whenever they reach the terminal state. After that, the values are discounted at every step from terminal to the initial state by both the algorithms. The states that are closer to the goal state have larger values. Therfore, if the path becomes too large, the states closer to the initial state will not be gathered by the algorithms and thus the steps will not be updated efficiently. The algorithm is not able to learn anything from the big frozen lake environment as all the values returned are zeros which are similar to the values of the first episode. \par
		
		Sarsa and Q-learning works effectively when the state space is not too large. When the state space is not too large, both the algorithms will be able to successfully construct a path using the (state,action) values, as the discounts affect the values of the states that are closer to the initial states considerably less.
	\end{enumerate}
\section{Result and conslusion}
There are a number of tabular and non-tabular algorithms for finding the optimal policy for the given environment. Each one has its advantage based on the environment.  For this assignment, we are using give environment small frozen lake and a big frozen lake. We can find the optimal policy for the small frozen lake environment using policy iteration, value iteration, Sarsa, Q-learning and linear approximation of Saras and Q-learning. To find the optimal policy of a big frozen lake we can use policy iteration value but Saras and Q-learning can not find the optimal policy for a big frozen lake because of the large state space.. 
\end{document}  